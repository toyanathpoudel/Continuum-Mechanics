
\documentclass[a4paper,12pt]{article}

% Language and encoding
\usepackage[utf8]{inputenc}       % UTF-8 input encoding
\usepackage[T1]{fontenc}          % Font encoding
\usepackage[english]{babel}       % Language support

% Math packages
\usepackage{amsmath, amssymb, amsfonts}  % Math environments, symbols, fonts
\usepackage{mathtools}            % Extensions to amsmath

% For boxed equations
\usepackage{empheq}               % Box or highlight equations easily

% For vectors and bold math symbols
\usepackage{bm}                  % Bold math symbols, e.g. \bm{a}

% For theorem-like environments (if needed)
\usepackage{amsthm}

% For graphics and color
\usepackage{graphicx}
\usepackage{xcolor}

% For tables and arrays
\usepackage{array}
\usepackage{booktabs}

% For bibliography and citations
\usepackage[numbers]{natbib}

% For glossary and nomenclature
\usepackage{nomencl}
\usepackage[acronym]{glossaries}

% For customizing table of contents etc.
\usepackage{tocloft}

% Dummy text (remove later)
\usepackage{lipsum}

% Theorem environment example (not used here but ready)
\newtheorem{theorem}{Theorem}[section]

\begin{document}


% Include the cover page (no page number)


%---------------------------
% Title page environment
%---------------------------


  \begin{center}
  
    \includegraphics[width=50mm]{tu.jpg}\\[0.8cm]

    \LARGE\textbf{TRIBHUVAN UNIVERSITY}\\[0.3cm]
    \Large\textbf{INSTITUTE OF ENGINEERING}\\[0.2cm]
    \textbf{Pulchowk Campus}\\[0.4cm]
    \vspace{0.3cm}

    A Report On\\
    \LARGE\textbf{Continuum Mechanics}\\[0.3cm]
    \large\textbf{Assignment No. : 01\\[0.1cm]
    Vectors and Tensors\\[0.1cm]}

    \vspace{1.2cm}

    \textbf{Submitted by:}\\
    \large{
      Name : Toyanath Poudel\\
      Roll no. : 081msmce023\\}

    \vspace{1.2cm}

    \textbf{Submitted to:}\\
    \textbf{
      Department of Mechanical and Aerospace Engineering\\
      Lecturer: Dr. Professor Mahesh Chandra Luintel
    }\\[0.2cm]
    {\large \today}

    \vspace*{\fill}
  \end{center}


\thispagestyle{empty} % Remove page number from cover
\clearpage


% Start page numbering from here
\pagenumbering{arabic}
\setcounter{page}{1}

\begin{center}
\section*{\MakeUppercase{Vectors and Tensors}}
\end{center}


\textbf{1. State which of the following expressions are meaningful in the suffix notation. Write out the unabridged versions of the meaningful expressions:}\\

\begin{enumerate}
    \item[a.] $a_{ii}$ - \textbf{\fbox{Meaningful}}\\
    
    \text{If an index repeats exactly twice in a single term, it implies summation over 
    that index.} \\
      \[
\sum_{i=1}^{3} a_{ii}
\]
    $=a_{11} + a_{22} + a_{33}$\\
    
    \item[b.] $a_{ij} b_j$ - \textbf {\fbox{Meaningful}} \\
    $=a_{i1}b_1 + a_{i2}b_2 + a_{i3}b_3$ \\
    $=a_{11}b_1 + a_{12}b_2 + a_{13}b_3$, \quad
    $a_{21}b_1 + a_{22}b_2 + a_{23}b_3$, \quad
    $a_{31}b_1 + a_{32}b_2 + a_{33}b_3$
    
    \item[c.] $a_{ij}b_i$ - \textbf{\fbox{Meaningful}} \\
    \text{The repeated index i appears exactly twice (once in each term)}\\
    $=a_{1j}b_1 + a_{2j}b_2 + a_{3j}b_3$ \\
    $=a_{11}b_1 + a_{21}b_2 + a_{31}b_3$, \quad
    $a_{12}b_1 + a_{22}b_2 + a_{32}b_3$, \quad
    $a_{13}b_1 + a_{23}b_2 + a_{33}b_3$\\
    
    \item[d.] $a_{ii}b_i$ - \textbf{\fbox{Not meaningful}},\quad\\ 
    \text{i appears three times total in the expression.}
    
    \item[e.] $a_{ii}b_{jj}$ - \textbf{\fbox{Meaningful}}\\
    $=a_{i1}b_{i1} + a_{i2}b_{i2} + a_{i3}b_{i3}$\\
    $=a_{11}b_{11} + a_{12}b_{12} + a_{13}b_{13} + a_{21}b_{21} + a_{22}b_{22} + a_{23}b_{23} + a_{31}b_{31} + a_{32}b_{32} + a_{33}b_{33}$\\
    
    \item[f.] $a_{ii}b_{ii}$ - \textbf{\fbox{Not Meaningful}}\\
    
    \item[g.] $a_{rs}b_{sr}$ - \textbf{\fbox{Meaningful}} \\
    $=a_{11}b_{11} + a_{12}b_{21} + a_{13}b_{31} + a_{21}b_{12} + a_{22}b_{22} + a_{23}b_{32} + a_{31}b_{13} + a_{32}b_{23} + a_{33}b_{33}$\\

    \item[h.] $a_{rs}b_{ss}$ - \textbf{\fbox{Not Meaningful}} \\
    
    \item[i.] $a_{ijk}b_{ik}$ - \textbf{\fbox{Meaningful}} \\
    $= a_{ij1}b_{i1} + a_{ij2}b_{i2} + a_{ij3}b_{i3}$ \\
    $= a_{1j1}b_{11} + a_{2j1}b_{21} + a_{3j1}b_{31} + a_{2j1}b_{21} + a_{2j2}b_{22} +         a_{2j3}b_{23} + a_{3j1}b_{31} + a_{3j2}b_{32} + a_{3j3}b_{33}$\\
    
    The three components are,\\
    $= a_{111}b_{11} + a_{211}b_{21} + a_{311}b_{31} + a_{211}b_{21} + a_{212}b_{22} +         a_{213}b_{23} + a_{311}b_{31} + a_{312}b_{32} + a_{313}b_{33}$,\\
    
    $ a_{121}b_{11} + a_{221}b_{21} + a_{321}b_{31} + a_{221}b_{21} + a_{222}b_{22} +         a_{223}b_{23} + a_{321}b_{31} + a_{322}b_{32} + a_{323}b_{33}$,\\
    
    $ a_{131}b_{11} + a_{231}b_{21} + a_{331}b_{31} + a_{231}b_{21} + a_{232}b_{22} +         a_{233}b_{23} + a_{331}b_{31} + a_{332}b_{32} + a_{333}b_{33}$\\
\end{enumerate}


\begin{enumerate}
\item[2.]{Write in indicial notation the equation:}
    \begin{itemize}
        \item[(a)] \( s = A_1^2 + A_2^2 + A_3^2 \)\\[6pt]
        \textbf{Ans:}
        \begin{align*}
            s &= A_1 \cdot A_1 + A_2 \cdot A_2 + A_3 \cdot A_3 \\
              &= A_i A_i
        \end{align*}
        
        \item[(b)] 
        \[
        \frac{\partial^2 \phi}{\partial x_1^2} + \frac{\partial ^2 \phi}{\partial x_2^2} + \frac{\partial^2 \phi}{\partial x_3^2}=0
        \]
        \textbf{Ans:}
        \[
          {\frac{\partial^2 \phi}{\partial x_1\cdot\partial x_1 } + \frac{\partial ^2 \phi}{\partial x_2\cdot \partial x_2} + \frac{\partial^2 \phi}{\partial x_3 \cdot \partial x_3}=0}\] \\ 
        \[{\frac{\partial^2 \phi}{\partial x_i \partial x_i}=0}
        \]
          
    \end{itemize}
\end{enumerate}


\textbf{3.Write the equation in Expanded form:}

\begin{enumerate}
     General form:
    \[
    a_i = \frac{\partial v_i}{\partial t} + v_j \cdot \frac{\partial v_i}{\partial x_j}
    \]

    Expanded form:
    \[
    a_i = \frac{\partial v_i}{\partial t} + v_1 \cdot \frac{\partial v_i}{\partial x_1} + v_2 \cdot \frac{\partial v_i}{\partial x_2} + v_3 \cdot \frac{\partial v_i}{\partial x_3}
    \]

    For \(i=1\):
    \[
    a_1 = \frac{\partial v_1}{\partial t} + v_1 \cdot \frac{\partial v_1}{\partial x_1} + v_2 \cdot \frac{\partial v_1}{\partial x_2} + v_3 \cdot \frac{\partial v_1}{\partial x_3}
    \]

   For \(i=2\):
    \[
    a_2 = \frac{\partial v_2}{\partial t} + v_1 \cdot \frac{\partial v_2}{\partial x_1} + v_2 \cdot \frac{\partial v_2}{\partial x_2} + v_3 \cdot \frac{\partial v_2}{\partial x_3}
    \]

     For \(i=3\):
    \[
    a_3 = \frac{\partial v_3}{\partial t} + v_1 \cdot \frac{\partial v_3}{\partial x_1} + v_2 \cdot \frac{\partial v_3}{\partial x_2} + v_3 \cdot \frac{\partial v_3}{\partial x_3}
    \]    
\end{enumerate}



\textbf{4.Write an indicia notation the matrix equation}

    \[[E] =[B]^T [C] [F]\]
    \\
    \textbf{Ans:}
    \[E_{ij}=B_{ip}^T C_{pk} F_{kj}\]
    \[E_{ij}=B_{pi} C_{pk} F_{kj}\]
    \\
    
\textbf{5.Write down the following equation in a condensed form:}
\begin{enumerate}
    \item[(a)] $ a_{11}=a_{22}=a_{33}=-p$\\
              $ a_{12}=a_{21}=a_{13}=a_{31}=a_{23}=a_{32}=0$\\
              \\
    \textbf{Ans:}\\
   We have the Kronecker delta defined as:\\
    \[
% Start of displayed math mode
\delta_{ij} = % Kronecker delta symbol with subscripts i and j
\begin{cases} % Begin a piecewise (cases) environment
1 & \text{if } i = j \\ % First case: 1 if i = j
0 & \text{if } i \ne j % Second case: 0 if i ≠ j
\end{cases} % End of cases environment
\] 


% If you want the formula to stand out, use display math (\[ ... \]).
%If you want it inside a sentence, use inline math ($ ... $).


Therefore, we can write:
    \[
    a_{ij} = -p \cdot \delta_{ij}
    \]



    \item[(b)] 
    $ a_{11}=\alpha (b_{11}+ b_{22}+ b_{33}) + \beta b_{11}$\\
    $ a_{22}=\alpha (b_{11}+ b_{22}+ b_{33} )+ \beta b_{22}$\\
    $ a_{33}=\alpha (b_{11}+ b_{22}+ b_{33} )+ \beta b_{33}$\\
    $ a_{12}=\beta b_{12},a_{21}=\beta b_{21},a_{23}=\beta b_{23}\\,a_{32}=\beta b_{32},a_{13}=\beta b_{13},a_{31}=\beta b_{31},$\\

    \textbf{Ans:}
\[
\begin{aligned}
a_{ii} &= \alpha (b_{11} + b_{22} + b_{33}) + \beta b_{ii} \quad \text{(for diagonal elements)} \\
a_{ij} &= \beta b_{ij} \quad \text{for } i \neq j \quad \text{(off-diagonal elements)}
\end{aligned}
\]

    Compact indicial notation:
    \[
    a_{ij} = \alpha \delta_{ij} \sum_{k=1}^3 b_{kk} + \beta b_{ij}
    \]

    Using Einstein summation convention:
    \[
    a_{ij} = \alpha \delta_{ij} b_{kk} + \beta b_{ij}
    \]
    where \( \delta_{ij} \) is the Kronecker delta and \( b_{kk} \) is the trace of \( b_{ij} \).
 \\     
  \\  

\textbf{6.Simplify the following:}


    \item[(a)] $\delta_{ij}(a_{ij} - a_{ji})$ \\ \textbf{Ans:}  \noindent
\makebox[\textwidth][1cm]{%
  \(
  \boxed{
    \delta_{ij} =
    \begin{cases}
      1 & \text{if } i = j \\
      0 & \text{if } i \ne j
    \end{cases}
  }
  \)
}
    If $i = j$:\\
    $ =\delta_{ii}(a_{ij} - a_{ji})$\\
    $= 1 \cdot (a_{ii} - a_{ii})\\ = 0$ \\ 
    
    If $i \ne j$: \\
    $=\delta_{ij}(a_{ij} - a_{ji})\\= 0 \cdot (a_{ij} - a_{ji})\\= 0$
    \\
    \item[(b)] $\delta_{ip} \delta_{jq} a_{p} b_{j} c_q$
    
    We have the relation $\boxed{\delta_{ip}\cdot a_{p}=a_{i}}$
    \\
    $\delta_{ip} \delta_{jq} a_{p} b_{j} c_{q}$ \\ 
    $= a_{i} b_{j} c_{j}$\\
    
   
 


 \item[(c)] $(\delta_{ij} + a_{ij})(\delta_{ij} - a_{ij})$ \\
    \noindent
\makebox[\textwidth][2.2cm]{%
  \boxed{%
    \begin{aligned}
      \delta_{ij} &= \delta_{ji} \\
      \delta_{kk} &= 3 \\
      a_i &= \delta_{ik} a_k \\
      A_{ij} &= \delta_{ik} A_{kj}
    \end{aligned}
  }
}
\textbf{Ans:}\quad
    If $i = j$: \\
    $(\delta_{ii} + a_{ii})(\delta_{ii} - a_{ii})\\ = \delta_{ii}^2 - a_{ii}^2\\ = 1 - a_{ii}^2$ \\
    \\Summed over $i$: \\

    $= \sum_{i} (1 - a_{ii}^2)\\ = 3 - a_{11}^2 - a_{22}^2 - a_{33}^2$
    \\
    
    If $i \ne j$: \\
    $(\delta_{ij} + a_{ij})(\delta_{ij} - a_{ij})\\ = \delta_{ij}^2 - a_{ij}^2 = 0 - a_{ij}^2 \\= -a_{ij}^2$ \\
    Summed over all $i \ne j$:  \\
    $= 3 - a_{ij}a_{ij}$
    \\

    \item[(d)] $\delta_{ik} \delta_{kp} \delta_{pi}$ \\
    $ =\delta_{ip}\cdot \delta_{pi}$\\
    $= \delta_{ii}\\
    =\delta_{11}+\delta_{22}+\delta_{33}$
    =3
    \\
%=====================================================
\\
\textbf{7.Show that}

\[
\varepsilon_{ijk} = 
\begin{vmatrix}
\delta_{i1} & \delta_{i2} & \delta_{i3} \\
\delta_{j1} & \delta_{j2} & \delta_{j3} \\
\delta_{k1} & \delta_{k2} & \delta_{k3}
\end{vmatrix}
\]

\textit{Indicial Proof:\\}
We know that,\

\begin{cases}
+1 & \text{if } (i,j,k) \text{ is an even permutation of } (1,2,3) \\
-1 & \text{if } (i,j,k) \text{ is an odd permutation of } (1,2,3) \\
0 & \text{if any two indices are equal}
\end{cases}\\



We use the identity:
\[
\varepsilon_{ijk} = \sum_{p=1}^{3} \sum_{q=1}^{3} \sum_{r=1}^{3} \varepsilon_{pqr} \delta_{ip} \delta_{jq} \delta_{kr}
\]

Using the sifting property of the Kronecker delta:
\[
\boxed{\delta_{ip} \varepsilon_{pqr} = \varepsilon_{i q r}, \quad
\delta_{jq} \varepsilon_{i q r} = \varepsilon_{i j r}, \quad
\delta_{kr} \varepsilon_{i j r} = \varepsilon_{ijk}}
\]

Therefore,
\[
\varepsilon_{ijk} = \sum_{pqr} \varepsilon_{pqr} \delta_{ip} \delta_{jq} \delta_{kr}
= \varepsilon_{ijk}
\]

So the determinant expression is equal to \(\varepsilon_{ijk}\), proving the identity.

\vspace{1em}

\textbf{(b)} Show that:
\[
\varepsilon_{ijk} a_i b_j = 
\begin{vmatrix}
a_1 & a_2 & a_3 \\
b_1 & b_2 & b_3 \\
\delta_{k1} & \delta_{k2} & \delta_{k3}
\end{vmatrix}
\]

\textit{Indicial Proof:}

Start from the determinant expression:
\[
\begin{vmatrix}
a_1 & a_2 & a_3 \\
b_1 & b_2 & b_3 \\
\delta_{k1} & \delta_{k2} & \delta_{k3}
\end{vmatrix}
= \sum_{p=1}^{3} \sum_{q=1}^{3} \sum_{r=1}^{3} \varepsilon_{pqr} a_p b_q \delta_{kr}
\]

By the sifting property of the delta:
\[
= \sum_{p=1}^{3} \sum_{q=1}^{3} \varepsilon_{pqk} a_p b_q
\]

Now relabel dummy indices \(p \to i\), \(q \to j\):
\[
= \varepsilon_{ijk} a_i b_j
\]

Therefore:
\[
\varepsilon_{ijk} a_i b_j = 
\begin{vmatrix}
a_1 & a_2 & a_3 \\
b_1 & b_2 & b_3 \\
\delta_{k1} & \delta_{k2} & \delta_{k3}
\end{vmatrix}
\]















\textbf{8.Show that}
Let \( \vec{v} = \vec{a} \times \vec{b}, \) Using indicial notation, show that:\\
\begin{itemize}
  \item[(a)] \( \vec{v} \cdot \vec{v} = a^2 b^2 \sin^2 \theta \)
  \item[(b)] \( \vec{a} \times \vec{b} \cdot \vec{a} = 0 \)
  \item[(c)] \( \vec{a} \times \vec{b} \cdot \vec{b} = 0 \)
\end{itemize}

\textbf{Solution:}

\textbf{(a)} For the given vector, we have:
\[
\begin{aligned}
\vec{v} \cdot \vec{v} 
&= \varepsilon_{ijk} a_j b_k \hat{e}_i \cdot \varepsilon_{pqs} a_q b_s \hat{e}_p \\
&= \varepsilon_{ijk} a_j b_k \varepsilon_{ipq} a_p b_q
\end{aligned}
\]

Using the identity:

\[
\boxed{
\varepsilon_{ijk} \varepsilon_{ipq}
= \delta_{jp} \delta_{kq} - \delta_{jq} \delta_{kp}
}
\]

We get, 
\[
\begin{aligned}
\vec{v} \cdot \vec{v}&= (\delta_{jp} \delta_{kq} - \delta_{jq} \delta_{kp}) a_j b_k a_p b_q \\
&= a_j b_k a_j b_k - a_j b_k a_k b_j\\
&= (\vec{a} \cdot \vec{a})(\vec{b} \cdot \vec{b}) - (\vec{a} \cdot \vec{b}) (\vec{a} \cdot \vec{b})\\
&= a^2 b^2 - (\vec a \cdot \vec b )^2\\
&=a^2 b^2 - (ab \cos \theta)^2\\
&= a^2 b^2 (1 - \cos^2 \theta)\\
&= a^2 b^2 \sin^2 \theta\\
\end{aligned}
\]

\textbf{(b)}
\begin{align*}
(\vec{a} \times \vec{b}) \cdot \vec{a} 
&= (\varepsilon_{ijk} a_j b_k \hat{e}_i) \cdot a_q \hat{e}_q \\
&= \varepsilon_{ijk} a_j b_k a_q \delta_{iq} \\
&= \varepsilon_{ijk} a_j b_k a_i \\
&= 0
\end{align*}


This is zero by symmetry in \( i \) and \( j \).\\ 

\textbf{(c)} :
\begin{align*}
(\vec{a} \times \vec{b}) \cdot \vec{b}
&= \varepsilon_{ijk} a_j b_k \hat{e}_i \cdot b_q \hat{e}_q \\
&= \varepsilon_{ijk} a_j b_k b_q \delta_{iq} \\
&= \varepsilon_{ijk} a_j b_k b_i \\
&= 0
\end{align*}


Again, this is zero by symmetry in \( i \) and \( k \).\\



%======================================================


\textbf{9.Verify the following Vector Identities using indicial notation}


\item[(a)]
\[
\vec{a} \times\vec{b} =-\vec{b} \times \vec{a}
\]\\




\textbf{Cross Product in Index Notation:}

\[
[\vec{a} \times \vec{b}]_i = \varepsilon_{ijk} a_j b_k
\]
\[
[\vec{b} \times \vec{a}]_i = \varepsilon_{ijk} b_j a_k
\]

\textbf{Demonstrating Antisymmetry:}

\[
\begin{aligned}
[\vec{a} \times \vec{b}]_i &= \varepsilon_{ijk} a_j b_k \\
&= -\varepsilon_{ikj} a_k b_j \quad \text{(antisymmetry: swap } j \leftrightarrow k) \\
&= -\varepsilon_{ijk} b_j a_k \quad \text{(rename dummy indices } j \leftrightarrow k) \\
&= -[\vec{b} \times \vec{a}]_i
\end{aligned}
\]

\textbf{Therefore:}

\[
\boxed{\vec{a} \times \vec{b} = -(\vec{b} \times \vec{a})}
\]




 



\item[(b)]
\[
\vec{a} \cdot (\vec{b} \times \vec{c}) = \vec{b} \cdot (\vec{c} \times \vec{a})
\]\\
\textbf{Ans:}\\

\[
\begin{aligned}
[\vec{a} \cdot (\vec{b} \times \vec{c})]_{i}
&=\vec{a}_i \cdot (\vec{b} \times \vec{c})_{i}\\
&=\varepsilon_{ijk} a_i  b_j  c_k\\
&= \varepsilon_{ijk} a_i b_j c_k
\end{aligned}
\]
\[
\begin{aligned}
[\vec{b} \cdot (\vec{c} \times \vec{a})]_i 
&=\vec{b}_i \cdot (\vec{c} \times \vec{a})_{i}\\
&= \varepsilon_{ijk} b_i c_j a_k\\
&= \varepsilon_{kij} a_i b_j c_k\\
&=\varepsilon_{ijk} a_i b_j c_k\\
&=[\vec{a} \cdot (\vec{b} \times \vec{c})]_{i}
\end{aligned}
\]

\textbf{ By using cyclic property, }
\[
\varepsilon_{kij} = \varepsilon_{ijk}
\Rightarrow \varepsilon_{kij} a_i b_j c_k = \varepsilon_{ijk} a_i b_j c_k
\]
\[
\boxed{
\vec{a} \cdot (\vec{b} \times \vec{c}) = 
\vec{b} \cdot (\vec{c} \times \vec{a}) = 
\vec{c} \cdot (\vec{a} \times \vec{b})
}
\]













\item[(c)] 
\[
\vec{a} \times (\vec{b} \times \vec{c}) = (\vec{a} \cdot \vec{c}) \vec{b} - (\vec{a} \cdot \vec{b}) \vec{c}.
\]

\vspace{10pt}

\textbf Express the left-hand side in index notation:
\[
\left[\vec{a} \times (\vec{b} \times \vec{c})\right]_i = \varepsilon_{ijk} a_j (\vec{b} \times \vec{c})_k = \varepsilon_{ijk} a_j \varepsilon_{klm} b_l c_m.
\]

\vspace{5pt}

\textbf Use the identity
\[\boxed{
\varepsilon_{ijk} \varepsilon_{klm} = \delta_{il} \delta_{jm} - \delta_{im} \delta_{jl}}.
\]


\textbf{Substitute and simplify:}
\[
\begin{aligned}
[\vec{a} \times (\vec{b} \times \vec{c})]_i 
&= (\delta_{il} \delta_{jm} - \delta_{im} \delta_{jl}) a_j b_l c_m \\
&= b_i a_j c_j - c_i a_j b_j \\
&= b_i (\vec{a} \cdot \vec{c}) - c_i (\vec{a} \cdot \vec{b}).
\end{aligned}
\]
\[
\boxed{
\vec{a} \times (\vec{b} \times \vec{c}) = (\vec{a} \cdot \vec{c}) \vec{b} - (\vec{a} \cdot \vec{b}) \vec{c}.
}
\]







\item[(d)]\[(\vec{a}\times \vec{b}) \cdot (\vec{c} \times \vec{d})] = (\vec{a} \cdot \vec{c}) \cdot (\vec{b}\cdot \vec{d}) - (\vec{a} \cdot \vec{d}) \cdot (\vec{b} \cdot \vec{c})
\]
\textbf{Ans:}\\
Let \(\vec{a}, \vec{b}, \vec{c}, \vec{d}\) be vectors


\textbf{Proof:}

\textbf Express the left-hand side in index notation:
\[
[(\vec{a} \times \vec{b}) \cdot (\vec{c} \times \vec{d})]_i = (\vec{a} \times \vec{b})_i (\vec{c} \times \vec{d})_i = \varepsilon_{ijk} a_j b_k \, \varepsilon_{imn} c_m d_n.
\]

\vspace{5pt}

\textbf Use the relation: 
\[\boxed{
\varepsilon_{ijk} \varepsilon_{imn} = \delta_{jm} \delta_{kn} - \delta_{jn} \delta_{km}},
\]

where \(\delta_{ij}\) is the Kronecker delta.

\vspace{5pt}

\textbf Substitute this identity:
\[
\begin{aligned}
(\vec{a} \times \vec{b}) \cdot (\vec{c} \times \vec{d}) 
&= (\delta_{jm} \delta_{kn} - \delta_{jn} \delta_{km}) a_j b_k c_m d_n \\
&= \delta_{jm} \delta_{kn} \, a_j b_k c_m d_n - \delta_{jn} \delta_{km} \, a_j b_k c_m d_n
\end{aligned}
\]



\vspace{5pt}

\textbf Apply the properties of the Kronecker delta:

\[
= a_m b_n c_m d_n - a_n b_m c_m d_n,
\]
\textbf{Since} \quad
\boxed{
\begin{aligned}
\delta_{jm} a_j &= a_m, \\
\delta_{kn} b_k &= b_n,
\end{aligned}
}
\quad \text{etc.}


\vspace{5pt}

\textbf Then:
\[
= (\vec{a} \cdot \vec{c})(\vec{b} \cdot \vec{d}) - (\vec{b} \cdot \vec{c})(\vec{a} \cdot \vec{d}).
\]

\vspace{10pt}

\textbf{Therefore,}
\[
\boxed{
(\vec{a} \times \vec{b}) \cdot (\vec{c} \times \vec{d}) = (\vec{a} \cdot \vec{c})(\vec{b} \cdot \vec{d}) - (\vec{b} \cdot \vec{c})(\vec{a} \cdot \vec{d}).
}
\]





\item[(e)]\[(\vec{a} \times \vec{b}) \times (\vec{c} \times \vec{d}) = (\vec{d} \cdot (\vec{a} \times \vec{b}))\, \vec{c} - (\vec{c} \cdot (\vec{a} \times \vec{b}))\, \vec{d}
\]
\textbf{Ans:}\\
Using indicial notation:
\[
[(\vec{a} \times \vec{b}) \times (\vec{c} \times \vec{d})]_i = \varepsilon_{ijk} (\vec{a} \times \vec{b})_j (\vec{c} \times \vec{d})_k
\]

We write the cross products as:
\[
(\vec{a} \times \vec{b})_j = \varepsilon_{jmn} a_m b_n, \quad (\vec{c} \times \vec{d})_k = \varepsilon_{kpq} c_p d_q
\]

So:
\[
[(\vec{a} \times \vec{b}) \times (\vec{c} \times \vec{d})]_i = \varepsilon_{ijk} \varepsilon_{jmn} \varepsilon_{kpq} a_m b_n c_p d_q
\]

Using the identity:
\[
\boxed{\varepsilon_{ijk} \varepsilon_{jmn} = \delta_{im} \delta_{kn} - \delta_{in} \delta_{km}}
\]

we get:
\[
= (\delta_{im} \delta_{kn} - \delta_{in} \delta_{km}) \varepsilon_{kpq} a_m b_n c_p d_q
\]

Expanding the terms:
\begin{align*}
&= \delta_{im} \delta_{kn} \varepsilon_{kpq} a_m b_n c_p d_q - \delta_{in} \delta_{km} \varepsilon_{kpq} a_m b_n c_p d_q \\
&= a_i b_k \varepsilon_{kpq} c_p d_q - b_i a_k \varepsilon_{kpq} c_p d_q \\
&= a_i (\vec{b} \cdot (\vec{c} \times \vec{d})) - b_i (\vec{a} \cdot (\vec{c} \times \vec{d}))
\end{align*}

Thus,
\[
[(\vec{a} \times \vec{b}) \times (\vec{c} \times \vec{d})]_i = (\vec{d} \cdot (\vec{a} \times \vec{b}))\, c_i - (\vec{c} \cdot (\vec{a} \times \vec{b}))\, d_i
\]

So in vector form:
\[
\boxed{
(\vec{a} \times \vec{b}) \times (\vec{c} \times \vec{d}) = (\vec{d} \cdot (\vec{a} \times \vec{b}))\, \vec{c} - (\vec{c} \cdot (\vec{a} \times \vec{b}))\, \vec{d}
}
\]

\vspace{4cm}
\begin{center}
\Huge \textbf{Thank You!}
\end{center}


\end{document}
